\begin{titlepage}
\newpage

\begin{center}
\bf
МІНІСТЕРСТВО ОСВІТИ І НАУКИ УКРАЇНИ \\
ОДЕСЬКИЙ НАЦІОНАЛЬНИЙ ПОЛІТЕХНІЧНИЙ УНІВЕРСИТЕТ\\
ХЕРСОНСЬКИЙ ПОЛІТЕХНІЧНИЙ КОЛЕДЖ\\
%Рада директорів ВНЗ І–ІІ р. а. Херсонської області\\
%Херсонський політехнічний коледж\\
%Одеського національного політехнічного університету\\
%(базовий ВНЗ І–ІІ р. а. Херсонської області)

\end{center}


\vspace{8em}

\begin{center}
\Large\bf 
МЕТОДИЧНІ ВКАЗІВКИ \\
ДО ВИКОНАННЯ КУРСОВОГО ПРОЕКТУ З ДИСЦИПЛІНИ\\
<<WEB-ПРОГРАМУВАННЯ>>\\
\vspace{1em}
для студентів спеціальності \\
122 <<Комп'ютерні науки та інформаційні технології>>\\
Галузь знань: 12  <<Інформаційні технології>>\\

\end{center}
\vspace{5em}

\begin{center}
Затверджено методичною радою ХПТК ОНПУ \\
Протокол №\_\_\_ від \_\_\_\_\_\_\_\_\_\_20\_\_\_ р. \\
\end{center}


\vspace{\fill}

\begin{center}
Херсон --- 2017
\end{center}

\end{titlepage}



% ненужное можно просто закомментировать знаком процента "%" 

% первую страницу не нумеруем
\thispagestyle{empty}			
Методичні вказівки до виконання курсового проекту  з дисципліни «Web-програмування» / Уклад.: В.М. Левицький~--– Херсон: ХПТК ОНПУ,~2017.~---– \_\_\_\_с.
\vspace{2em}

\begin{center}
Навчальне видання\\
{\bf
МЕТОДИЧНІ ВКАЗІВКИ\\
ДО ВИКОНАННЯ КУРСОВОГО ПРОЕКТУ\\
ДИСЦИПЛІНИ <<WEB-ПРОГРАМУВАННЯ>>\\
для спеціальності\\ 
122 <<Комп'ютерні науки та інформаційні технології>>\\
Галузь знань: 12  <<Інформаційні технології>>\\
}
\end{center}
\vspace{2em}
Укладач		     
\hfill
Левицький Віктор Миколайович, викладач \\
\begin{center}
\end{center}
%\vspace{1em}
Відповідальний \hfill \\
редактор              \\          
Технічний \hfill \\
редактор     \\
Коректор      \hfill \\                 
Рецензент         \hfill \\                                                   
\begin{center}
\end{center}
\vspace{1em}
За редакцією укладачів \\
Надруковано з оригінал-макета замовника \\


% название
%\title{Методичні вказівки до лабораторних робіт з дисципліни <<web-програмування>>}
%\author{Жарікова М.В., Левицький В.М.}
%\maketitle

% печатаем содержание
%\chapter*[Зміст]{Зміст}
\tableofcontents


