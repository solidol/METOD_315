\chapter{Загальні відомості до виконання та захисту курсової роботи}
Тематика курсових робіт розглядається і схвалюється на засіданні циклової комісії, погоджується  завідувачем відділення, та затверджується заступником директора.
Перед початком роботи над курсовою роботою студент повинен отримати заповнений, підписаний, ухвалений та затверджений у встановленому порядку аркуш «ЗАВДАННЯ» та індивідуальний план, розроблений керівником проекту. В бланку <<ЗАВДАННЯ>> наведено:
\begin{itemize}
\item тему курсової роботи;
\item дані до виконання роботи;
\item вказівки по змісту пояснювальної записки;
\item дата видачі завдання до курсової роботи, термін закінчення роботи студента над курсовою роботою;
\item список джерел, що рекомендуються;
\item індивідуальний план виконання курсової роботи з нормованою кількістю балів.
\end{itemize}

Бланк завдання на курсову роботу підшивається студентом до пояснювальної записки після листа затвердження.

Курсова робота виконується студентом самостійно при консультуванні його керівником проекту у відповідності з графіком. Кожен студент повинен відвідувати консультації згідно з графіком. При відсутності студента на двох або більше консультаціях, керівник подає доповідну записку завідувачу відділення.

На всьму протязі курсового проектування керівник оцінює якість виконання етапів КР у балах рейтингової системи, проставляє їх у відповідний графах індивідуального плану.

Попередній захист курсової роботи відбувається в установлений термін. Мета попереднього захисту – визначення рівня готовності студента до захисту проекту. На попередній захист студент надає програмний додаток, пояснювальну записку (не переплетену). На попередньому захисті програмний додаток повинен знаходитись на етапі тестування, виконувати всі функції згідно поставленої задачі.
Після перевірки працездатності програмного додатку та готовності пояснювальної записки керівник повертає курсову роботу (програму та пояснювальну записку) для ознайомлення із зауваженнями та вказівками щодо виправлення помилок.

Виправлена пояснювальна записка із підписами студента та керівника проекту подається на нормоконтроль, який здійснюється  за відповідним графіком в два етапи:

\begin{itemize}
\item на першому етапі — проводиться перевірка текстової документації із розстановкою олівцем позначок нормоконтролера щодо порушень вимог Єдиної системи програмної документації (ЄСПД) і повертаються студентові для усунення порушень,
\item на другому етапі — здійснюється перевірка виправлених помилок, допущених студентом у пояснювальній записці, оцінювання виконання курсового проекту на відповідність вимог стандартів ЄСПД та СОУ 2:2008,  усунення позначок нормоконролера. При наявності невиправлених порушень нормоконтролер проставляє позначки червоною пастою і підписує документ.
Максимальний бал, який студент може отримати за сумарним рейтингом складає 60 балів. Такий бал, за бажанням студента, може бути зарахованим як оцінка знань без захисту проекту. Сума балів від 36 до 60 є допуском до захисту.
\end{itemize}

До захисту курсової роботи студент готує доповідь (обсягом до 10 хвилин), в якій чітко формулює постановку задачі курсової роботи, пояснює послідовність її виконання. Доповідає про результати роботи, вказує на можливості реального використання курсової роботи.
Захист курсової роботи здійснюється у присутності комісії у складі двох-трьох кваліфікованих викладачів за участю керівника курсового проекту.
При оцінюванні результатів захисту КР студент може отримати максимально 40 балів, при цьому враховується :
\begin{itemize}
\item оригінальність, самостійність розробки функціональної моделі та методів реалізації;
\item використання у програмному продукті CASE-технологій;
\item навички використання UML-мови моделювання;
\item використання технологій проектування ПЗ;
\item грамотність написання та оформлення пояснювальної записки відповідно до вимог ЄСПД;
\item ступінь використання довідкової та технічної літератури, ДСТУ, ГОСТ;
\item вміння грамотно захищати розроблену роботу.
\end{itemize}

Студент, який при захисті курсової роботи отримав незадовільну оцінку, допускається, як виняток, до повторного захисту у новий термін, але після позитивного рішення циклової комісії та дозволу завідувача відділенням.

Студентам, які не захищали курсову роботу з поважної причини, яка документально підтверджена, завідувачем відділення може бути продовжено строк виконання та захисту.
 
 

