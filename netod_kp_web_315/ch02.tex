\chapter{Оформлення пояснювальної записки}
Пояснювальна записка до курсової  роботи виконується як комп’ютерний набір тексту або рукописним способом на одній стороні аркуша формату А4 ( 210 х 297 мм) відповідно до ГОСТ 2.301-68.

При комп’ютерному наборі необхідно виконувати оформлення курсової роботи на друкуючих та графічних пристроях виведення персональних комп’ютерів /ГОСТ 2.004-88/ з додержуванням правил Єдиної системи конструкторської документації та Єдиної системи програмної документації.

Пояснювальна записка виконується державною мовою.
Рекомендований обсяг пояснювальної записки повинен складати 25-30 сторінок друкованого тексту (комп’ютерний набір тексту виконується шрифтом гарнітури “Ariаl”, кегль - 14 пт, курсивом, з полуторним інтервалом).  В дану кількість сторінок не включають сторінки, на яких розміщено додатки.

З використанням  комп’ютерного способу виконання на сторінці має розташовуватися не більше 40 рядків (відповідно інтервал між рядками – не більше 8 мм).

Помилки, описки та графічні неточності допускається виправляти підчищенням або зафарбовуванням білою фарбою і нанесенням на тому ж місці виправленого зображення від руки. Виправлене повинно бути чорного кольору.

Заголовки   структурних елементів  і розділів слід розташовувати з абзацного відступу і друкувати малими  літерами, починаючи з першої великої літери без крапки в кінці, не підкреслюючи їх. Кожен структурний елемент та розділ слід починати з нової сторінки.

Заголовки підрозділів, пунктів, підпунктів слід починати з абзацного відступу і друкувати маленькими літерами, окрім першої великої, без крапки і не підкреслюючи їх. Абзацний відступ повинен бути однаковим впродовж усього тексту звіту і дорівнювати 5 знакам (15-17 мм). Перенесення слів у заголовку не допускається.

Відстань між заголовком і подальшим чи попереднім текстом за комп’ютерного  способу виконання – не менше, ніж 2 рядки.  Відстань між основами рядків заголовку, а також між двома заголовками приймають такою, як у тексті.

Не допускається розміщувати назви розділу, підрозділу, пункту, підпункту в нижній частині сторінки, якщо після неї розміщено тільки один рядок тексту.

Сторінки пояснювальної записки  нумеруються арабськими цифрами, додержуючись наскрізної нумерації і впродовж усього тексту. Номер сторінки проставляється у графі 7 основного напису відповідно до ДСТУ ГОСТ 2.104:2006.

Оформлювальний аркуш титульного листа з темою і підписами, лист завдання  включають до загальної нумерації сторінок звіту, але номер не проставляють. Елементи дати проставляють повно (наприклад, 21.11.2017).  Зразок оформлення титульного аркуша затвердження наводиться у додатку А. 

Структурні елементи «РЕФЕРАТ», «ЗМІСТ», «ВСТУП», «ВИСНОВКИ», «РЕКОМЕНДАЦІЇ», «СПИСОК ВИКОРИСТАНИХ ДЖЕРЕЛ» не нумеруються, а їх назви правлять за назви структурних елементів.
Обсяг пояснювальної записки курсової роботи не повинен перебільшувати 25 – 30 аркушів без врахування лістингу програми.
Структура пояснювальної записки:

\begin{itemize}
\item лист затвердження;
\item завдання на курсову роботу;
\item зміст;
\item розділи пояснювальної записки;
\item висновки;
\item список використаних джерел;
\item додатки.
\end{itemize}


Пояснювальна записка повинна бути виконана у відповідності до стандартів ЄСКД та ЄСПД, діючих на даний час:
\begin{itemize}
\item ДСТУ 3008-2015 Інформація та документація. Звіти у сфері науки і техніки. Структура і правила оформлення.
\item ГОСТ 2.004-88 ЕСКД. Общие требования к вьполнению конструкторских и технологических документов на печатающих и графических устройствах вывода ЭВМ. ЗАЧЕМ???? СТП 17.2-2002 МНОЖИТ НА 0 ЭТОТ ГОСТ!!!
\item ГОСТ 2.301-68 ЕСКД. Форматы.
\item ГОСТ 19.106-78 ЕСПД. Требования к программным документам, вьполненным печатным способом.
\item ГОСТ 19.701-90 ИСО 5807-85. Схемы алгоритмов, программ, данных и систем. Условные обозначения и правила выполнения
\item ГОСТ 2.105-95 ЕСКД. Общие требования к текстовым документам.
\item СТП 17.2-2002 Правила оформлення пояснювальної записки до дипломних та курсових робіт. Загальні відомості.
\end{itemize}


