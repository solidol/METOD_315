\chapter{Робота з файлами та каталогами}
\section*{Мета роботи}
Навчитися створювати на диску файли, проводити читання та запис у них. Освоїти переміщення по файлу, копіювання, переміщення, блокування та видалення. Навчитися завантажувати та зберігати файл на сервері.

\section{Поняття файлу, операції з файлами}
\index{Файл}
\index{Файл!функції}
\textbf{Файл}~--- блок інформації на зовнішньому запам'ятовуючому пристрої комп'ютера, що має певне логічне подання, відповідні йому операції читання-запису і, як правило, фіксоване ім'я. 

Основними операціями для роботи з файлами є 
\begin{itemize}
\item створення
\item читання
\item запис
\item переміщення покажчика
\item копіювання
\item переміщення
\item блокування
\item видалення.
\end{itemize}

Робота з файлами реалізована у РНР для всіх підтримуваних платформ, але слід зауважити, що у шляхах до файлів використовується прямий слеш <<\verb'/'>>. для повноцінної роботи з файлами у додатку~\ref{fs-func:table} дано повний перелік перелік функцій, нижче більш детально розглянуто базові функції.
\section{Створення, відкриття та закриття файлу}
\index{Файл!створення}
\index{Файл!відкриття/закриття}
PHP надає доступ до файлів в операційних системах Windows і Unix для читання, запису або додавання вмісту. 

PHP містить функції \verb'fopen()' і \verb'fclose()' для роботи з файлами. Опис функцій надано нижче.

\verb|fopen(filename, mode [, use_include_path [, zcontext]])|~--- функція використовується для відкриття файлу. Функції потрібно задати ім'я файлу і режим роботи. Вона повертає покажчик на файл, що використовується в якості посилання. Якщо немає можливості відкрити/створити файл функція повертає значення \verb'FALSE'. Якщо \verb'filename' передане у формі <<\verb|scheme://...|>>, воно вважається URL'ом і PHP проведе пошук обробника даного протоколу. Список параметрів \verb'mode', що визначають, яким чином буде проводитись робота з дано у таблиці~\ref{fo-par:table} додатків.

\verb|fclose(handle)|~--- функція використовується для закриття файлу. Функції потрібно задати ідентифікатор файлу, створений при відкритті файлу за допомогою функції \verb|fopen()|. Повертає \verb'TRUE' при успішній роботі або \verb'FALSE' при відмові.

\section{Блочні читання та запис}
\index{Файл!блочні читання/запис}
Блочне зчитування та запис у мові <<РНР>> подібні до мови <<C>>. 

Для зчитування використовується функція \verb'fread(handle, length)', яка повертає послідовність байтів розміром \verb'length', що прочитана з відкритого файлу.

Блочний запис проводиться за допомогою функції \verb'fwrite(handle, string [, length])', що записує вміст \verb'string' у файловий потік \verb'handle'. Якщо переданий аргумент \verb'length', запис зупиниться після того, як \verb'length' байтів будуть записані або буде досягнутий кінець рядка \verb'string', дивлячись що станеться першим.

\verb'fwrite()' повертає кількість записаних байтів або \verb'FALSE' в разі помилки. 




\verb'fgetcsv(handle [,length [,delimiter [,enclosure]]]))'~--- функція, яка використовується для читання вмісту файлу і аналізу даних для створення масиву. Дані поділяються параметром-обмежувачем, заданим у функції. Ця функція читає вміст файлу і створює масив, роблячи доступними певні частини тексту.



\section{Порядкові читання та запис}
\index{Файл!порядкові читання/запис}

\verb'fgets(handle [, length])'~--- функція повертає рядок розміром в length-1 байт, прочитану з дескриптора файлу, на який вказує параметр \verb'handle'. Читання закінчується, коли кількість прочитаних байтів досягає length-1, по досягненні кінця рядка або по досягненні кінця файлу. Якщо довжина не вказана, за замовчуванням її значення дорівнює 1 кілобайт або 1024 байтам.

У разі виникнення помилки функція повертає FALSE. 

Для спрощеного порядкового запису можна використовувати функцію \verb'fputs(handle, string [, length])', яка є псевдонімом функції \verb'fwrite()'.

\section{Переміщення покажчика по файлу}
\index{Файл!переміщення покажчика}
Для переміщення покажчика у файлі використовується функція \verb'fseek(handle, offset [, whence])'. Механізм переміщення базується на встановленні зміщення у файлі, на який посилається \verb'handle'. Нове зміщення, вимірюване в байтах від початку файлу, знаходиться шляхом додавання параметра \verb'offset' до позиції, зазначеної в параметрі \verb'whence', значення якого визначаються таким чином:
\begin{itemize}
\item \verb'SEEK_SET'~--- Встановлює зсув в \verb'offset' байт.
\item \verb'SEEK_CUR'~--- Встановлює зсув в поточне плюс \verb'offset'.
\item \verb'SEEK_END'~--- Встановлює зсув в розмір файлу плюс \verb'offset'.
\end{itemize}

Якщо \verb'whence' не зазначений, за замовчуванням він встановлюється в \verb'SEEK_SET'.

У разі успіху повертає 0, в іншому випадку повертає -1. Зверніть увагу, що перехід до зміщення за кінцем файлу не вважається помилкою.

У РНР існує функція для перевірки досягнення кінця файлу \verb'feof(file)', функція повертає \verb'TRUE' у разі досягнення кінця файлу, або \verb'FALSE' якщо покажчик знаходиться всередині. Приклад використання функції надано нижче:

\begin{verbatim}
<?php
$f=fopen("myfile.txt","r");
while(!feof($f))
{ 
$st=fgets($f);
// обробка рядка $st
// . . .
}
fclose($f);
?>
\end{verbatim}
В даному прикладі проводиться порядкове зчитування до досягнення покажчиком кінця файлу.

Щоб дізнатись в якій позиції знаходиться покажчик у даний момент використовується функція \verb'ftell(file)'.
\section{Копіювання, переміщення та видалення файлу}
\index{Файл!копіювання}

Функція \verb'copy(source, dest)' створює копію файлу, чиє ім'я передано в параметрі \verb'source', у файлі з ім'ям \verb'dest'. Повертає \verb'TRUE' у разі успішного завершення або \verb'FALSE' в разі виникнення помилки.

Наступний приклад показує, як скопіювати вміст одного файлу в іншій файл:


\begin{lstlisting}[caption=Копіювання змісту файлу]
<?php
$orig_filename = "C:/Documents and Settings/
	Administrator/MyFiles/myfile.txt";
$new_filename = "C:/Documents and Settings/
	Administrator/MyFiles/myNewfile.txt";
$status = copy($orig_filename, $new_filename) 
	or die("Неможливо скопіювати файл");
echo "Файл скопійовано!";
?>
\end{lstlisting}




\index{Файл!переміщення}
Переміщення (або перейменування) файлу здійснюється функцією \verb'rename(oldname, newname [, context])'. Функція перейменовує файл \verb'oldname' на \verb'newname' та повертає значення \verb'TRUE' або \verb'FALSE' у випадку невдачі.

Наступний приклад показує, як перейменувати файл за допомогою функції \verb'rename()':



\begin{lstlisting}[caption=Перейменування файлу]
<?php
$orig_filename = "C:/Documents and Settings/
   Administrator/MyFiles/myfile.txt";
$new_filename = "C:/Documents and Settings/
   Administrator/MyFiles/newfile.txt";
$status = rename($orig_filename, $new_filename) 
   or exit("Неможливо перейменувати файл");

echo "файл успішно перейменовано";
?> 
\end{lstlisting}

\index{Файл!видалення}
Для видалення файлу з носія використовується функція \verb'unlink(filename)'. У разі використання операційних систем сімейства Unix видалення буде успішним коли число жорстких посилань на файл буде дорівнювати нулю.

\section{Блокування файлу}
\index{Файл!блокування}
У разі інтенсивного використання веб-додатком операцій читання-запису файлів та великій кількості користувачів додатку постає питання розділення доступу до файлів, до яких звертається програма.

У цьому випадку необхідно використовувати функцію блокування файлу \verb'flock(file, operation [, wouldblock])'. \textbf{Блокування файлу}~--- встановлення для зазначеного відкритого дескриптора файлу file режиму монопольного доступу, який би хотів отримати поточний процес. Цей режим задається аргументом \verb'operation' і може бути однією з наступних констант:
\begin{enumerate}
\item \verb'LOCK_SH' (або 1)~--- Колективне блокування;
\item \verb'LOCK_EX' (або 2)~--- виняткове блокування;
\item \verb'LOCK_UN' (або 3)~--- зняти блокування;
\item \verb'LOCK_NB' (або 4)~--- цю константу треба додати до однієї з попередніх, якщо ви не хочете, щоб програма <<підвисає>> на \verb'flock()' в очікуванні своєї черги, а відразу повертала управління.
\end{enumerate}
У випадку, якщо була вимога режиму без очікування, і блокування не було успішно встановлено, в необов'язковий параметр-змінну \verb'wouldblock' буде записано значення істина \verb'TRUE'.

У випадку помилки функція, як завжди, повертає \verb'FALSE', а в разі успішного завершення~--- \verb'TRUE'.


\subsection*{Виняткове блокування}
Якщо процесу необхідно монополізувати доступ до файлу, необхідно викликати функцію  \verb'flock(file, LOCK_EX)'. У цьому випадку він може бути абсолютно впевнений, що всі інші процеси не почнуть без дозволу писати у файл, поки він не виконає всі свої дії і не викличе \verb'flock(file, LOCK_UN)' або не закриє файл.

У випадку, коли поточний процес не єдиний, що потребує монопольного доступу до файлу, і доступ вже монополізовано, операційна система призупинить його виконання на функції \verb'flock()' і поставить його в чергу на доступ до файлу. Модель роботи з винятковим блокуванням надано нижче:


\begin{lstlisting}[caption=Виняткове блокування файлу]
<?
$f=fopen($f,"a+") or die("Не можу відкрити файл на запис!");
flock($f,LOCK_EX); // чекаємо, доки не буде отримане 
//виняткове блокування
// . . .
fflush($f); // записуємо всі зміни на диск
flock($f,LOCK_UN); // знімаємо блокування
fclose($f);
?>
\end{lstlisting}

\subsection*{Колективне блокування}
Колективне блокування вигідно використовувати у випадку, коли два або більше процесів здійснюють читання, або один записує, а всі інші читають. Суть у тому, що монополізація доступу до файлу здійснюється коли процес активний. Якщо колективне блокування увімкнено, але дій з файлом не проводиться, то доступ до нього можуть отримати інші процеси.



\begin{lstlisting}[caption=Колективне блокування файлу]
<?
$f=fopen($f,"r") or die("Не можу відкрити файл для читання");
flock($f,LOCK_SH); 
// Якщо інші процеси не пишуть у файл,
// то тут можна з нього читати
flock($f,LOCK_UN); // Знімаємо	блокування
fclose($f);
?>
\end{lstlisting}

\section{Робота з каталогами}
\index{Каталог}
\index{Каталог!функції}
\textbf{Каталог}~--- об'єкт в файловій системі, що спрощує організацію файлів. Типова файлова система містить велику кількість файлів, і каталоги допомагають упорядкувати її шляхом їх ієрархічного угруповання.

Повноцінна робота з каталогами реалізована у РНР для всіх платформ, але слід зауважити, що для функцій роботи з ними використовується UNIX-подібне написання слешу~--- <<\verb'/'>>. Робота з каталогами у РНР організована за допомогою декількох функцій, що виконують ряд базових операцій:
\begin{itemize}
\item відкриття/закриття каталогу
\item зміна поточного каталогу
\item створення каталогу
\item видалення каталогу
\item читання змісту каталогу та інші.

\end{itemize}
Список функції дано у додатку~\ref{dir-func:table}.

Нижче надано приклад обробки вмісту каталогу:

\begin{lstlisting}[caption=Обробка вмісту каталогу]
<?
$i = 0;
$handle = opendir ('D:/dir/');
while($file = readdir($handle))
{
  if ($file != '.' && $file != '..' )
  {
    $func[$i] = $file;
    $i++;
  }
}
sort ($func);
echo '<table>';
for ($q = 0; $q<sizeof($func); $q++)
   {
   echo '<tr>';
   echo '<td>.$heds[$q].'</td>'."\n";
   echo '</tr>'."\n";
   }
echo '</table>';
?>
\end{lstlisting}

Сценарій відкриває каталог \verb'D:\dir\', поелементно обходить його, та друкує відсортований зміст.

\section{Завантаження файлу на сервер}
\index{Файл!завантаження}
\index{HTML!форми!input}
Завантаження фаил на сервер здійснюється користувачами мережі інтернет доволі часто, а саме:
\begin{itemize}
\item Веб-ітерфейси поштових сервісів, які дозволяють додавати до листа додаток (attach), а для цього потрібно спочатку завантажити файл на сервер, і тільки після цього його можна додавати до листа;
\item Інтерактивні фотогалереї та фотоальбоми, які не можуть існувати без механізму завантаження файлів на сервер;
\item Портали безкоштовного програмного забезпечення, які використовують для обміну файлами різних програм, і.т.д. 
\end{itemize}


Завантаження файлу на сервер здійснюється за допомогою multipart-форми, в якій є поле завантаження файлу. В якості параметра enctype вказується значення multipart / form-data:
\begin{verbatim}
<form action="upload.php" method="post" enctype="multipart/form-data">
<input type="file" name="uploadfile">
<input type="submit" value="Загрузить">
</form>
\end{verbatim}

Ось так приблизно виглядатиме наведена \verb'multipart'-форма (Рис.~\ref{f5-1:image}):

\begin{figure}
\includegraphics[scale=1,width=8cm]{ch05-01.png}
\caption{Форма завантаження файлу}
\label{f5-1:image}
\end{figure}

Перш, ніж приступити до написання сценарію обробки \verb'multipart'-форми, потрібно відредагувати файл конфігурації \verb'php.ini', щоб дозволити завантаження файлів на сервер.

Конфігураційний файл PHP php.ini має три параметри, пов'язані з завантаженням файлів на сервер:
\begin{enumerate}
\item \verb'file_uploads = On'~--- дозволяє завантаження файлів на сервер по протоколу HTTP;
\item \verb'upoad_tmp_dir = /tmp'~--- встановлює каталог для тимчасового зберігання завантажених файлів;
\item \verb'upload_max_filesize = 2M'~--- встановлює максимальний обсяг завантажуваних файлів. 

\end{enumerate}

Для успішного завантаження та збереження файлу необхідно створити сценарій виду:


\begin{lstlisting}[caption=Завантаження файлу на сервер]
<html>
<head>
  <title> Результат завантаження файлу </title>
</head>
<body>
<?php
if($_FILES["filename"]["size"] > 1024*3*1024)
   {
   echo ("Розмір файлу перевищує три мегабайта");
   exit;
   }
// Перевіряємо	чи	завантажився	файл
if(is_uploaded_file($_FILES["filename"]["tmp_name"]))
   {
   // Якщо так, то переносимо його у кінцевий каталог
   move_uploaded_file($_FILES["filename"]["tmp_name"], 
   "/path/to/file/".$_FILES["filename"]["name"]);
   } 
else 
   {
   echo("Помилка завантаження");
   }
?>
</body></html>
\end{lstlisting}

Слід зауважити, що для доступу до файлу використовується суперглобальний масив \verb'$_FILES', детальний опис якого дано у додатку~\ref{sup-glob:app}.



 
\pagebreak[3]
\section{Індивідуальне завдання}

\nopagebreak[4]
\subsection*{Завдання до лабораторної роботи}
\nopagebreak[4]
\begin{enumerate}
\item Вивчити теоретичний матеріал
\item Відповісти на контрольні запитання
\item Скласти алгоритм (блок-схему) програми
\item Виконати практичне завдання
\item Скласти звіт
\item Захистити роботу
\end{enumerate}

\subsection*{Контрольні запитання}
\nopagebreak[4]
\begin{enumerate}
\item Що таке файл? Що таке каталог?
\item Які основні операції для роботи з файлами ви знаєте?
\item Яким чином можна відкрити/створити файл у РНР?
\item Яким чином проводиться читання з файлу?
\item Яким чином проводиться запис у файл?
\item Яким чином можна переміщувати покажчик у файлі?
\item Що таке блокування файлу?
\item Які види блокування файлів ви знаєте?
\item Які особливості видів блокування файлів?
\item Яким чином завантажити файл на сервер?

\end{enumerate}

\subsection*{Практичні завдання}
\nopagebreak[4]


\begin{enumerate}
\item[]Створити форму з текстовим полем вводу для шляху каталогу та вивести таку інформацію: 
\item список об'єктів каталогу, підкаталоги виділити іншим кольором
\item вивести тільки файли, та реалізувати можливість їх видалення
\item знайти підкаталоги першого рівня та вивести їхній зміст
\item знайти порожні підкаталоги першого рівня, вивести їхні імена та видалити їх
\item створити 10 підкаталогів, та по 10 файлів у кожному, вивести отримане дерево
\item []Написати просту гостьову книгу, що зберігає інформацію у файлах, яка буде записувати та виводити:
\item перші 10 повідомлень користувачів та час відправлення
\item перші 10 повідомлень користувачів та поставлений ними рейтинг від 1 до 10
\item 10 повідомлень з найвищим рейтингом
\item список повідомлень з можливістю їх видалення
\item список повідомлень з можливістю їх редагування
\item[]Продемонструвати роботу гостьової книги 
\item[]Створити книгу рецептів, серед функцій якої є: 
\item виведення рецепту з найвищим рейтингом
\item виведення найсвіжіших рецептів
\item виведення найперших рецептів
\item збереження рецептів в окремих файлах
\item видалення вибраного рецепту
\item редагування вибраного рецепту
\item[]Продемонструвати роботу книги рецептів 
\item[]Створити фотогалерею з функціями додавання фотографії та опису до неї та додатковими функціями: 
\item виведення посилань на фото
\item виведення посилань на опис
\item виведення списку фото та видалення обраного
\item виведення списку та редагування опису обраного
\item обмежити розмір файлу, що завантажується до 1МБ
\item[]Продемонструвати роботу фотогалереї  
\item[]Створити файлове сховище з можливістю завантаження файлу та опису до нього та додатковими функціями:
\item вивести опис файлу та посилання для завантаження
\item реалізувати можливість видалення файлу
\item реалізувати можливість редагування імені файлу
\item реалізувати можливість редагування опису
\item[]Продемонструвати роботу файлового сховища 
\end{enumerate}

